\documentclass[11pt,a4paper]{report}
\usepackage[latin1]{inputenc}
\usepackage[spanish]{babel}
\usepackage{amsmath}
\usepackage{amsfonts}
\usepackage{amssymb}
\usepackage{graphicx}
\usepackage{hyperref}
\usepackage[left=2cm,right=2cm,top=2cm,bottom=2cm]{geometry}

\author{Miguel Ignacio S�nchez }
\title{\textbf{Anexos PROYECTO TURING}}


\begin{document}
%\maketitle

\chapter*{Anexo I: Configurar Qtcreator}
Con este anexo quiero dejar claro como tengo que configurar Qtcreator para ide de desarrollo para ROS.

\begin{enumerate}


\item Tenemos que abrir Qtcreator desde terminal, por que autom�ticamente carga el fichero $~/.bashrc$ el cual es el encargado de configurar el workspace de ROS.
\item Una vez abierto el IDE de programaci�n Qtcreator los siguiente es abrir el proyecto CMakelist de orden mas superior, es decir, $~/catkin_ws/src/CMakelist$. Este es el truco, este enlaza directamente con toda la instalaci�n de ROS.
\item Despu�s de abrir el proyecto, configuramos el directorio build a \url{$catkin_ws/build/$} 
\item en los argumentos de Cmake ponemos: $-DCMAKE_INSTALL_PREFIX=../install$ \\
$-DCATKIN_DEVEL_PREFIX=../devel$ 
\item Hacemos click derecho sobre el directorio del proyecto y hacemos run CMake.  

\end{enumerate}
Provamos $https://answers.ros.org/question/91516/qt-creator-281-ros-and-catkin-problems/$





\chapter*{Anexo II: Configurar Qtcreator}






\end{document}